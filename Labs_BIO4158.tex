% Options for packages loaded elsewhere
\PassOptionsToPackage{unicode}{hyperref}
\PassOptionsToPackage{hyphens}{url}
%
\documentclass[
  12pt,
]{book}
\usepackage{lmodern}
\usepackage{amssymb,amsmath}
\usepackage{ifxetex,ifluatex}
\ifnum 0\ifxetex 1\fi\ifluatex 1\fi=0 % if pdftex
  \usepackage[T1]{fontenc}
  \usepackage[utf8]{inputenc}
  \usepackage{textcomp} % provide euro and other symbols
\else % if luatex or xetex
  \usepackage{unicode-math}
  \defaultfontfeatures{Scale=MatchLowercase}
  \defaultfontfeatures[\rmfamily]{Ligatures=TeX,Scale=1}
\fi
% Use upquote if available, for straight quotes in verbatim environments
\IfFileExists{upquote.sty}{\usepackage{upquote}}{}
\IfFileExists{microtype.sty}{% use microtype if available
  \usepackage[]{microtype}
  \UseMicrotypeSet[protrusion]{basicmath} % disable protrusion for tt fonts
}{}
\makeatletter
\@ifundefined{KOMAClassName}{% if non-KOMA class
  \IfFileExists{parskip.sty}{%
    \usepackage{parskip}
  }{% else
    \setlength{\parindent}{0pt}
    \setlength{\parskip}{6pt plus 2pt minus 1pt}}
}{% if KOMA class
  \KOMAoptions{parskip=half}}
\makeatother
\usepackage{xcolor}
\IfFileExists{xurl.sty}{\usepackage{xurl}}{} % add URL line breaks if available
\IfFileExists{bookmark.sty}{\usepackage{bookmark}}{\usepackage{hyperref}}
\hypersetup{
  pdftitle={BIO4158 Applied biostats with R},
  pdfauthor={Julien Martin},
  hidelinks,
  pdfcreator={LaTeX via pandoc}}
\urlstyle{same} % disable monospaced font for URLs
\usepackage{color}
\usepackage{fancyvrb}
\newcommand{\VerbBar}{|}
\newcommand{\VERB}{\Verb[commandchars=\\\{\}]}
\DefineVerbatimEnvironment{Highlighting}{Verbatim}{commandchars=\\\{\}}
% Add ',fontsize=\small' for more characters per line
\usepackage{framed}
\definecolor{shadecolor}{RGB}{248,248,248}
\newenvironment{Shaded}{\begin{snugshade}}{\end{snugshade}}
\newcommand{\AlertTok}[1]{\textcolor[rgb]{0.94,0.16,0.16}{#1}}
\newcommand{\AnnotationTok}[1]{\textcolor[rgb]{0.56,0.35,0.01}{\textbf{\textit{#1}}}}
\newcommand{\AttributeTok}[1]{\textcolor[rgb]{0.77,0.63,0.00}{#1}}
\newcommand{\BaseNTok}[1]{\textcolor[rgb]{0.00,0.00,0.81}{#1}}
\newcommand{\BuiltInTok}[1]{#1}
\newcommand{\CharTok}[1]{\textcolor[rgb]{0.31,0.60,0.02}{#1}}
\newcommand{\CommentTok}[1]{\textcolor[rgb]{0.56,0.35,0.01}{\textit{#1}}}
\newcommand{\CommentVarTok}[1]{\textcolor[rgb]{0.56,0.35,0.01}{\textbf{\textit{#1}}}}
\newcommand{\ConstantTok}[1]{\textcolor[rgb]{0.00,0.00,0.00}{#1}}
\newcommand{\ControlFlowTok}[1]{\textcolor[rgb]{0.13,0.29,0.53}{\textbf{#1}}}
\newcommand{\DataTypeTok}[1]{\textcolor[rgb]{0.13,0.29,0.53}{#1}}
\newcommand{\DecValTok}[1]{\textcolor[rgb]{0.00,0.00,0.81}{#1}}
\newcommand{\DocumentationTok}[1]{\textcolor[rgb]{0.56,0.35,0.01}{\textbf{\textit{#1}}}}
\newcommand{\ErrorTok}[1]{\textcolor[rgb]{0.64,0.00,0.00}{\textbf{#1}}}
\newcommand{\ExtensionTok}[1]{#1}
\newcommand{\FloatTok}[1]{\textcolor[rgb]{0.00,0.00,0.81}{#1}}
\newcommand{\FunctionTok}[1]{\textcolor[rgb]{0.00,0.00,0.00}{#1}}
\newcommand{\ImportTok}[1]{#1}
\newcommand{\InformationTok}[1]{\textcolor[rgb]{0.56,0.35,0.01}{\textbf{\textit{#1}}}}
\newcommand{\KeywordTok}[1]{\textcolor[rgb]{0.13,0.29,0.53}{\textbf{#1}}}
\newcommand{\NormalTok}[1]{#1}
\newcommand{\OperatorTok}[1]{\textcolor[rgb]{0.81,0.36,0.00}{\textbf{#1}}}
\newcommand{\OtherTok}[1]{\textcolor[rgb]{0.56,0.35,0.01}{#1}}
\newcommand{\PreprocessorTok}[1]{\textcolor[rgb]{0.56,0.35,0.01}{\textit{#1}}}
\newcommand{\RegionMarkerTok}[1]{#1}
\newcommand{\SpecialCharTok}[1]{\textcolor[rgb]{0.00,0.00,0.00}{#1}}
\newcommand{\SpecialStringTok}[1]{\textcolor[rgb]{0.31,0.60,0.02}{#1}}
\newcommand{\StringTok}[1]{\textcolor[rgb]{0.31,0.60,0.02}{#1}}
\newcommand{\VariableTok}[1]{\textcolor[rgb]{0.00,0.00,0.00}{#1}}
\newcommand{\VerbatimStringTok}[1]{\textcolor[rgb]{0.31,0.60,0.02}{#1}}
\newcommand{\WarningTok}[1]{\textcolor[rgb]{0.56,0.35,0.01}{\textbf{\textit{#1}}}}
\usepackage{longtable,booktabs}
% Correct order of tables after \paragraph or \subparagraph
\usepackage{etoolbox}
\makeatletter
\patchcmd\longtable{\par}{\if@noskipsec\mbox{}\fi\par}{}{}
\makeatother
% Allow footnotes in longtable head/foot
\IfFileExists{footnotehyper.sty}{\usepackage{footnotehyper}}{\usepackage{footnote}}
\makesavenoteenv{longtable}
\usepackage{graphicx}
\makeatletter
\def\maxwidth{\ifdim\Gin@nat@width>\linewidth\linewidth\else\Gin@nat@width\fi}
\def\maxheight{\ifdim\Gin@nat@height>\textheight\textheight\else\Gin@nat@height\fi}
\makeatother
% Scale images if necessary, so that they will not overflow the page
% margins by default, and it is still possible to overwrite the defaults
% using explicit options in \includegraphics[width, height, ...]{}
\setkeys{Gin}{width=\maxwidth,height=\maxheight,keepaspectratio}
% Set default figure placement to htbp
\makeatletter
\def\fps@figure{htbp}
\makeatother
\setlength{\emergencystretch}{3em} % prevent overfull lines
\providecommand{\tightlist}{%
  \setlength{\itemsep}{0pt}\setlength{\parskip}{0pt}}
\setcounter{secnumdepth}{5}
%\usepackage{booktabs}
\usepackage{ctable}
\usepackage{fancyhdr}
\usepackage{float}
\usepackage[margin=2cm]{geometry}

\floatplacement{figure}{H}

%\usepackage[sf,bf]{titlesec}

\hypersetup{colorlinks=true, urlcolor=blue}

\renewcommand{\chaptername}{Chapitre}
\renewcommand{\contentsname}{Table des Matières}
\renewcommand{\partname}{Partie}

\usepackage{framed,color}
\definecolor{incolor}{RGB}{240,240,240}
\definecolor{outcolor}{RGB}{248,248,248}

\renewcommand{\textfraction}{0.05}
\renewcommand{\topfraction}{0.8}
\renewcommand{\bottomfraction}{0.8}
\renewcommand{\floatpagefraction}{0.75}

%\renewenvironment{quote}{\begin{VF}}{\end{VF}}

\ifxetex
 \usepackage{letltxmacro}
 \setlength{\XeTeXLinkMargin}{1pt}
 \LetLtxMacro\SavedIncludeGraphics\includegraphics
 \def\includegraphics#1#{% #1 catches optional stuff (star/opt. arg.)
   \IncludeGraphicsAux{#1}%
 }%
 \newcommand*{\IncludeGraphicsAux}[2]{%
   \XeTeXLinkBox{%
     \SavedIncludeGraphics#1{#2}%
   }%
 }%
\fi

\makeatletter
\newenvironment{kframe}{%
\medskip{}
\setlength{\fboxsep}{.8em}
\def\at@end@of@kframe{}%
\ifinner\ifhmode%
 \def\at@end@of@kframe{\end{minipage}}%
 \begin{minipage}{\columnwidth}%
\fi\fi%
\def\FrameCommand##1{\hskip\@totalleftmargin \hskip-\fboxsep
\colorbox{incolor}{##1}\hskip-\fboxsep
    % There is no \\@totalrightmargin, so:
    \hskip-\linewidth \hskip-\@totalleftmargin \hskip\columnwidth}%
\MakeFramed {\advance\hsize-\width
  \@totalleftmargin\z@ \linewidth\hsize
  \@setminipage}}%
{\par\unskip\endMakeFramed%
\at@end@of@kframe}
\makeatother

\makeatletter
\@ifundefined{Shaded}{
}{\renewenvironment{Shaded}{\begin{kframe}}{\end{kframe}}}
\makeatother

% \let\oldverbatim\verbatim
% \renewenvironment{Shaded}{\vspace{0.2cm}\begin{kframe}}{\end{kframe}}
% \renewenvironment{verbatim}{\begin{shaded}\begin{oldverbatim}}{\end{oldverbatim}\end{shaded}}

\newenvironment{rmdblock}[1]
 {
 \begin{itemize}
 \renewcommand{\labelitemi}{
   \raisebox{-.7\height}[0pt][0pt]{
     {\setkeys{Gin}{width=3em,keepaspectratio}\includegraphics{images/#1}}
   }
 }
 \begin{kframe}
 \setlength{\fboxsep}{1em}
 \item
 }
 {
 \end{kframe}
 \end{itemize}
 }
\newenvironment{rmdnote}
  {\begin{rmdblock}{note}}
  {\end{rmdblock}}
\newenvironment{rmdcaution}
  {\begin{rmdblock}{caution}}
  {\end{rmdblock}}
\newenvironment{rmdimportant}
  {\begin{rmdblock}{important}}
  {\end{rmdblock}}
\newenvironment{rmdtip}
  {\begin{rmdblock}{tip}}
  {\end{rmdblock}}
\newenvironment{rmdwarning}
  {\begin{rmdblock}{warning}}
  {\end{rmdblock}}
\newenvironment{rmdcode}
  {\begin{rmdblock}{screen}}
  {\end{rmdblock}}

\usepackage{makeidx}
\makeindex

\urlstyle{tt}

\usepackage{amsthm}
\makeatletter
\def\thm@space@setup{%
  \thm@preskip=8pt plus 2pt minus 4pt
  \thm@postskip=\thm@preskip
}
\makeatother

% \frontmatter
\usepackage[]{natbib}
\bibliographystyle{plainnat}

\title{BIO4158 Applied biostats with R}
\usepackage{etoolbox}
\makeatletter
\providecommand{\subtitle}[1]{% add subtitle to \maketitle
  \apptocmd{\@title}{\par {\large #1 \par}}{}{}
}
\makeatother
\subtitle{Laboratory manual}
\author{Julien Martin}
\date{07-09-2021}

\begin{document}
\maketitle

%\cleardoublepage\newpage\thispagestyle{empty}\null
%\cleardoublepage\newpage\thispagestyle{empty}\null
%\cleardoublepage\newpage
%\thispagestyle{empty}
%\begin{center}
%\includegraphics{images/missing.png}
%\end{center}

%\setlength{\abovedisplayskip}{-5pt}
%\setlength{\abovedisplayshortskip}{-5pt}

{
\setcounter{tocdepth}{1}
\tableofcontents
}
\hypertarget{note}{%
\chapter*{Note}\label{note}}
\addcontentsline{toc}{chapter}{Note}

Development version. Lab material will appear slowly during the Fall 2021 term.

\hypertarget{preface}{%
\chapter*{Preface}\label{preface}}
\addcontentsline{toc}{chapter}{Preface}

The laboratory exercises outlined in the following pages are designed to allow you to develop some expertise in using statistical software (R)
to analyze data. R is powerful statistical software but, like all software, it has its limitations. In particular, it is dumb: it cannot think for you,
it cannot tell you whether the analysis you are attempting to do is appropriate or even makes any sense, and it cannot interpret your
results.

\hypertarget{general-points-to-keep-in-mind}{%
\section*{General points to keep in mind}\label{general-points-to-keep-in-mind}}
\addcontentsline{toc}{section}{General points to keep in mind}

\begin{itemize}
\item
  Before attempting any statistical procedure, you must familiarize yourself with what the procedure is actually doing. This does not mean you actually have to know the underlying mathematics (although this certainly helps!), but you should at least understand the principles involved in the analysis. Therefore, before doing a laboratory exercise, read the appropriate section(s) in the lecture notes. Otherwise, the output from your analyses - even if done correctly - will seem like drivel.
\item
  The laboratories are designed to complement the lectures, and vice versa. Owing to scheduling constraints, it may not be possible to synchronize the two perfectly. But feel free to bring questions about the laboratories to class, or questions about the lectures to the labs.
\item
  Work on the laboratories at your own speed: some can be donemuch more quickly than others, and one laboratory need not correspond to one laboratory session. In fact, for some laboratorieswe have allotted two laboratory sessions. Although you will notbe ``graded'' on the laboratories per se, be aware that completingthe labs is essential. If you do not complete the labs, it is veryunlikely that you will be able to complete the assignments and thefinal exam/term paper. So take these laboratories seriously!
\item
  The objective of the first lab is to allow you to acquire or reviewthe minimum knowledge required to complete the following laboratory exercises with R. There are always several methods toaccomplish something in R, but you will only find simple ways inthis manual. Those amongst you that want to go further will easilyfind many examples of more detailed and sophisticated methods.In particular, I point you to the following resources:

  \begin{itemize}
  \tightlist
  \item
    R for beginners
    \url{http://cran.r-project.org/doc/contrib/Paradis-rdebuts_en.pdf}
  \item
    An introduction to R
    \url{http://cran.r-project.org/doc/manuals/R-intro.html}
  \item
    If you prefer paper books, the CRAN web site has a commented list at :
    \url{http://www.r-project.org/doc/bib/R-books.html}
  \item
    Excellent list of R books
    \url{https://www.bigbookofr.com/}
  \item
    R reference card by Tom Short
    \url{http://cran.r-project.org/doc/contrib/Short-refcard.pdf}
  \end{itemize}
\end{itemize}

\hypertarget{what-is-r-and-why-use-it-in-this-course}{%
\section*{What is R and why use it in this course?}\label{what-is-r-and-why-use-it-in-this-course}}
\addcontentsline{toc}{section}{What is R and why use it in this course?}

R is multiplatform free software forming a system for statistical computation and graphics. R is also a programming language specially designed for statistical data analysis. It is a dialect of the S language. S- Plus is another dialect of the S language, very similar to R, incorporated into a commercial package. S-Plus has a built-in graphical design intreface that some find convivial.

R has 2 major advantages for this course. Initially, you will find that it also has one inconvenience. However, this ``inconvenience'' will rapidly force you to acquire very good working habits. So, I see it as a third advantage.

The first advantage is that you can install it freely on you personal computer(s). This is important because it is by doing analyses that you will learn and eventually master biostatistics. This implies that you have easy and unlimited access to a statistical software package. The second advantage is that R can do everything in statistics. R was conceived to be extensible and has become the preferred tool for statisticians around the world. The question is not ``Can R do this?'' but rather ``How can I do this in R?''. And search engine are your friends.

No other software package offers you these two advantages.

The inconvenience of R is that one has to type commands (or copy and paste code) rather than use a menu and select options. If you do not know what command to use, nothing will happen. It is therefore not that easy when you start. However, it is possible to rapidly learn to make basic operations (open a data file, plot data, and run a simple analysis). And once you understand the operating principle, you can easily find examples on the Web for more complex analyses and graphs for which you can adapt the code.

This is exactly what you will do in the first lab to familiarize yourself with R.

Why is this inconvenience really an advantage in my mind? Because this way of doing things is more efficient and will save you time on the long run. I guarantee it. Believe me, you will never do an analysis only once. As you'll proceed through analyses, you will find data entry errors, discover that the analysis must be run separately for subgroups, find extra data, have to rerun the analysis on transformed data, or you will make some analytical error along the way. If you use a graphical interface with menus, redoing an analysis implies that you reclick here, enter values there, select some options, etc. Each of these steps is a potential source of error. If, instead, you use lines of codes, you only have to fix the code and submit to repeat instantaneously the entire analysis. And you can perfectly document what you did, leaving an audit trail for the future. This is how pros work and can document the quality of the results of their analyses.

\hypertarget{software-installation}{%
\section*{Software installation}\label{software-installation}}
\addcontentsline{toc}{section}{Software installation}

\hypertarget{r}{%
\subsection*{R}\label{r}}
\addcontentsline{toc}{subsection}{R}

To install R on a computer, go to \url{http://cran.r-project.org/}. You will find compiled versions (binairies) for your preferred operating system (Windows, MacOS, Linux).

Note : R has already been installed on the lab computers (the version may be slightly different, but this should not matter).

\hypertarget{rstudio-or-vs-code}{%
\subsection*{Rstudio or VS code}\label{rstudio-or-vs-code}}
\addcontentsline{toc}{subsection}{Rstudio or VS code}

RStudio and VS code are integrated development environment software or IDE. RStudio was develop specifically to work with R. VScode is more generela but work extremely well with R. Both are available on Windows, OS X and Linux

\begin{itemize}
\tightlist
\item
  RStudio: \url{https://www.rstudio.com/products/rstudio/download/}
\item
  VScode: \url{https://code.visualstudio.com/download}
\end{itemize}

\hypertarget{r-libraries}{%
\subsection*{R libraries}\label{r-libraries}}
\addcontentsline{toc}{subsection}{R libraries}

R is essentially unlimited in terms of functions that can be used, because is relies on functions packages that can be added as extra components to use in R.

\begin{itemize}
\tightlist
\item
  Rmarkdown
\item
  tinytex
\end{itemize}

Those 2 packages should be installed automatically with RStudio but I recommend to install them manually in case they are not. To do so, just copy-paste the text below in R terminal.

\begin{Shaded}
\begin{Highlighting}[]
\KeywordTok{install.packages}\NormalTok{(}\KeywordTok{c}\NormalTok{(}\StringTok{"rmarkdown"}\NormalTok{, }\StringTok{"tinytex"}\NormalTok{))}
\end{Highlighting}
\end{Shaded}

\hypertarget{gpower}{%
\subsection*{G*Power}\label{gpower}}
\addcontentsline{toc}{subsection}{G*Power}

G*Power est un programme gratuit, développé par des psychologues de l'Université de Dusseldorf en Allemagne.
Le programme existe en version Mac et Windows.
Il peut cependant être utilisé sous linux via Wine.
G*Power vous permettra d'effectuer une analyse de puissance pour la majorité des tests que nous verrons au cours de la session sans avoir à effectuer des calculs complexes ou farfouiller dans des tableaux ou des figures décrivant des distributions ou des courbes de puissance.

Téléchargez le programme sur le site \url{https://www.psychologie.hhu.de/arbeitsgruppen/allgemeine-psychologie-und-arbeitspsychologie/gpower.html}

\hypertarget{general-laboratory-instructions}{%
\section*{General laboratory instructions}\label{general-laboratory-instructions}}
\addcontentsline{toc}{section}{General laboratory instructions}

\begin{itemize}
\tightlist
\item
  Bring a USB key or equivalent so you can save your work. Alternatively, email your results to yourself.
\item
  Read the lab exercise before coming to the lab. Read the R code and come with questions about the code.
\item
  During pre-labs, listen to the special instructions
\item
  Do the laboratory exercises at your own rhythm, in teams. Then, I
  recommend that you start (complete?) the lab assignment so that you can benefit from the presence of the TA or prof.
\item
  During your analyses, copy and paste results in a separate document, for example in your preferred word processing program. Annotate abundantly
\item
  Each time you shut down R, save the history of your commands (ex: labo1.1 rHistory, labo1.2.rHistory, etc). You will be able to redo the lab rapidly, get code fragments, or more easily identify errors.
\item
  Create your own ``library'' of code fragments (snippets). Annotate
  it abundantly. You will thank yourself later.
\end{itemize}

\hypertarget{notes-about-the-manual}{%
\section*{Notes about the manual}\label{notes-about-the-manual}}
\addcontentsline{toc}{section}{Notes about the manual}

You will find explanations on the theory, R code and functions, IDE best practice and exercises with R.

The manual tries to highlight some part of the text using the following boxes and icons.

\begin{rmdcode}
Exercises,
\end{rmdcode}

\begin{rmdcaution}
warnings,
\end{rmdcaution}

\begin{rmdwarning}
warnings,
\end{rmdwarning}

\begin{rmdimportant}
important points
\end{rmdimportant}

\begin{rmdnote}
notes
\end{rmdnote}

\begin{rmdtip}
and tips
\end{rmdtip}
\#\#\# Resources \{-\}

This document was developped using the excellent \href{https://bookdown.org/}{bookdown} 📦 de \href{https://yihui.name/}{Yihui Xie}. The manual is based on the previous lab manual \emph{Findlay, Morin and Rundle, BIO4158 Laboratory manual for BIO4158}.

\hypertarget{license}{%
\subsection*{License}\label{license}}
\addcontentsline{toc}{subsection}{License}

The document is available follwoing the license \href{http://creativecommons.org/licenses/by-nc-sa/4.0/}{License Creative Commons Attribution-NonCommercial-ShareAlike 4.0 International}.

\begin{figure}
\centering
\includegraphics{images/icons/license_cc.png}
\caption{License Creative Commons}
\end{figure}

\printindex

\end{document}
